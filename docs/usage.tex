\section{Usage}

This section explains a little of the ways you can use Tempus to view the current time.

Tempus runs as an ordinary desktop application, using familiar menus, windows, and controls. It has one window, displaying the time in a scalable view as the central information in big type.
You run Tempus as you would any other program, by double-clicking its icon or running it from the command line. 

The location obviously affects the time, so Tempus needs to know your location to calculate sidereal times and other things. You can set your location in the \texttt{Location} menu.

\subsection{Formats and standards}

The current time formats supported are:
\begin{itemize}
    \item Local Mean Time, 12 hour
    \item Local Mean Time, 24 hour
    \item Universal Time
    \item Local Sidereal Time
    \item Greenwich Sidereal Time
\end{itemize}

City coordinates are of course points within an urban area, and therefore not more precise than so many other points. The current values are taken from the \texttt{dateandtime.info} web site. Currently, only the time zone values are of use for the display of time.

Tempus aims to adhere to the following scientific standards:

\begin{itemize}
    \item World Geodetic System (WGS84, EPSG:4326) - in the \texttt{locations.xml} file, for example
\end{itemize}

